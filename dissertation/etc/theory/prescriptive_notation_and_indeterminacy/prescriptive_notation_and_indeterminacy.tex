\documentclass[11pt]{article}
\usepackage{geometry}
\geometry{letterpaper}
\usepackage[parfill]{parskip}
\usepackage{graphicx}
\usepackage{amssymb}
\usepackage{epstopdf}
\DeclareGraphicsRule{.tif}{png}{.png}{`convert #1 `dirname #1`/`basename #1 .tif`.png}

\title{Prescriptive Notation and Performative Indeterminacy in \textit{Viscera}}
\author{Joseph Davancens}
\date{December 17, 2015}

\begin{document}
\maketitle

\section{Introduction}
\subsection{Prescriptive Notation}
In my dissertation composition, which has the working title \textit{Viscera}, I employ a specialized notation system as means to achieve certain compositional ends, such as single-instrument polyphony and the elicitation of unstable instrumental sounds with a strong noise component.  The notation is prescriptive, i.e., it specifies what kind of actions an instrumentalist should perform. This is contrasted with traditional musical notation, which specifies what kind of sounds the instrumentalist should make.

One of the main precepts of this piece is separating the various sound-producing mechanisms (embouchure, bowing, fingering, etc.) into distinct, rhythmically independent streams of musical activity. This fosters the emergence of unusual, unexpected, and sometimes unpredictable sonic phenomena from standard orchestral instruments.

\subsection{Specifics of Notation}
The notation system uses a combination of woodwind fingering tablature, traditional rhythmic notation, and two-dimensional vector art representing instrument parameter changes. Each instrument in the ensemble is assigned a staff group. The content of each staff group varies depending on the instrument family. Woodwind staff groups have an embouchure parameter staff and left and right hand fingering tablature staves. Each of these three staves is paired with a percussion-like staff indicating the rhythm of the prescribed actions. The piano staves are standard. Stringed instrument staff groups contain one large staff representing the vertical string-space of the instrument, from the nut to the bridge, and two rhythm staves for bowing and fingering actions. Bow and fingering parameters are inscribed in the string-space staff.

\section{History and precedents}
\subsection{Indeterminacy in Cage, Xenakis, Stockhausen, Wolff, Brown}
Composers have exploited notions of randomness and unpredictability throughout the latter half of the twentieth century. Indeterminancy is located at various stages in the transmission of a work from composer to listener. In some pieces, chance is used in the composition of the piece itself. In John Cage's \textit{Music of Changes} (1951) for solo piano, many score parameters like pitch, duration, and dynamics were derived by consulting the I-Ching. In Iannis Xenakis' \textit{Pithopratka} (1955-56) for string orchestra, the composer used probablilty distributions to control the temporal density of a large number of sound events. Different performances of the same work are identical if not similar.

Other pieces locate indeterminacy in performance. In open-form pieces like Stockhausen's \textit{Klavierstück XI} (1956), and Christian Wolff's \textit {Duo for Pianists II} (1958), score parameters are fixed and rationally determined, but the order in which sections of the score are performed is left to the performer. This ordering is intended to be decided upon at the moment of performance. In these works the material is consistent between performances, but the variabilty of section order reconfigures the flow of the work between performances, creating new relationships and continuities among successive materials.

Another kind of intdeterminacy involves a performer's interpretation on non-traditional graphical elements in a score. In Earle Brown's \textit {December 1952} (1952), for solo piano, rectangles of various lengths and widths give some indication of pitch range, temporal density, and sounding duration of piano clusters, but the specfic choices as to which keys are hit when, how hard, and held for how long are left to the pianist to be decided upon either before or during a performance. This provides and extremely wide range of variability between performances.


\subsection{Hubler's Radical Instrumentalism}



\section{My Composition Methods}
\subsection{The Action-Maker-Handler paradigm}

\section{Indeterminacy in \textit{Viscera}}
\subsection{Imprecision in interpretation}
\subsection{Inherent indeterminacy}
\subsection{Indeterminate sounds in otherwise deteriminate musical contexts}

\section{Conclusion}
\end{document}
