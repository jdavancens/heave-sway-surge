\documentclass[11pt]{article}
\usepackage{geometry}
\geometry{letterpaper}
\usepackage[parfill]{parskip}
\usepackage{graphicx}
\usepackage{amssymb}
\usepackage{epstopdf}
\DeclareGraphicsRule{.tif}{png}{.png}{`convert #1 `dirname #1`/`basename #1 .tif`.png}

\title{Prescriptive Notation and Performative Indeterminacy in \textit{Viscera}}
\author{Joseph Davancens}
\date{December 17, 2015}

\begin{document}
\maketitle

\section{Introduction}
\subsection{Prescriptive Notation}
In my dissertation composition, which has the working title \textit{Viscera}, I employ a specialized notation system as means to achieve certain compositional ends, such as single-instrument polyphony and the elicitation of unstable instrumental sounds with a strong noise component.  The notation is prescriptive, i.e., it specifies what kind of actions an instrumentalist should perform. This is contrasted with traditional musical notation, which specifies what kind of sounds the instrumentalist should make.

One of the main precepts of this work is separating the various sound-producing mechanisms (embouchure, bowing, fingering, etc.) into distinct, rhythmically independent streams of musical activity. This fosters the emergence of unusual, unexpected, and sometimes unpredictable sonic phenomena from standard orchestral instruments.

\subsection{Specifics of Notation}
The notation system uses a combination of woodwind fingering tablature, traditional rhythmic notation, and two-dimensional vector art representing instrument parameter changes. Each instrument in the ensemble is assigned a staff group. The content of each staff group varies depending on the instrument family. Woodwind staff groups have an embouchure parameter staff and left and right hand fingering tablature staves. Each of these three staves is paired with a percussion-like staff indicating the rhythm of the prescribed actions. The piano staves are standard. Stringed instrument staff groups contain one large staff representing the vertical string-space of the instrument, from the nut to the bridge, and two rhythm staves for bowing and fingering actions. Bow and fingering parameters are inscribed in the string-space staff.

\section{History and precedents}
\subsection{Indeterminacy in Cage, Xenakis, Stockhausen, Wolff, Brown}
Composers have exploited notions of randomness and unpredictability throughout the latter half of the twentieth century. Indeterminancy is located at various stages in the transmission of a work from composer to listener. In some pieces, chance is used in the composition of the piece itself. In John Cage's \textit{Music of Changes} (1951) for solo piano, many score parameters like pitch, duration, and dynamics were derived by consulting the I-Ching. In Iannis Xenakis' \textit{Pithopratka} (1955-56) for string orchestra, the composer used probablilty distributions to control the temporal density of a large number of sound events. Different performances of the same work are identical if not similar.

Other pieces locate indeterminacy in performance. In open-form pieces like Stockhausen's \textit{Klavierstück XI} (1956), and Christian Wolff's \textit {Duo for Pianists II} (1958), score parameters are fixed and rationally determined, but the order in which sections of the score are performed is left to the performer. This ordering is intended to be decided upon at the moment of performance. In these works the material is consistent between performances, but the variabilty of section order reconfigures the flow of the work between performances, creating new relationships and continuities among successive materials.

Another kind of intdeterminacy involves a performer's interpretation on non-traditional graphical elements in a score. In Earle Brown's \textit {December 1952} (1952), for solo piano, rectangles of various lengths and widths give some indication of pitch range, temporal density, and sounding duration of piano clusters, but the specfic choices as to which keys are hit when, how hard, and held for how long are left to the pianist to be decided upon either before or during a performance. This provides and extremely wide range of variability between performances.

\subsection{Hubler's Radical Instrumentalism}
Klaus K. Hubler developed an approach to instrumental writing in the 1970s that treated separately the "activators" of an instrumental techniques (such as the bow and left hand fingers of a string player). In Hubler's works, performance parameters take precedence over pitch as defining structural elements. In \textit {Cercar} (1983) for solo trombone, there are staves that indicate the harmonic generated by lip pressure, air pressure and slide position.
\ldots

\section{My Composition Methods}
\subsection{The Action-Maker-Handler paradigm}
I've designed software to facilitate the composition and typesetting of \textit {Viscera}. It takes the form of a collection of scripts, data models and classes written with and around the Python package Abjad. Abjad models musical score structure and creates Lilypond input files. The central concept of \textit {Viscera's} software component is what I am calling an "Action-Maker-Handler" paradigm. With this paradigm, the musical work consists of a set of outside-score materials in the form of data strutures repesenting instrumental actions (Action), a choreographing process that orders and "rhythmicizes" these action-materials (Maker), and a process that translates this choreography to score notation (Handler).
\ldots

\subsection{Materials and Form}
\ldots

\subsection{Rhythm}
\ldots

\section{Indeterminacy in \textit{Viscera}}
\subsection{Acoustical indeterminacy}
The way that I chose to handle performance parameters in \textit {Viscera} allows for the potential for unpredicitabilty in sounds that result from a performer's interpretation of the score during a performance . This manifests in a few ways, depending on the instruments involved.  For examples, the interaction of air support, embouchure pressures, and fingerings on woodwind instruments is predictable when specified according to established performance practices. Trained woodwind performers fine-tune these parameters in order to create an optimal tone (i.e. stable, in-tune) best suited for a given musical scenario. But when these parameters are treated as raw data, without a clear indication of the aural result, performance of these parameters can result in less predictable and less stable ways, such as squeaks, variations in tone-color, microtonal inflection and multiphonics. Likewise, with string instruments, the interaction of bow pressure, speed, position on string, left hand finger pressure and position can produce unstable sounds like flickering harmonics, squeaks and crunching noises. The approach I've taken is intended to open up the soundworld of the work to include these kinds of sounds.

\subsection{Imprecision in interpretation}
Apart from acoustical indeterminacy, unpredictabilty can result from the reading of parameters specified in the score. Much of the notation communicates continuous parameters in spatial way, as opposed to discrete parameters like loudness represented symbolically in traditional notation. On the string staves in \textit {Viscera}, bow and finger positions along the length of the string aresrepresented by the height of a line and pressure is indicated by the darkness of the line. While these parameters are explicitly and precisely determined, they can only be approximated by the performer in their interpretation. The mind is better at perceiving relative differences than precise values.

\subsection{The role of indeterminacy in \textit {Viscera}}
Indeterminacy in \textit {Viscera} serves to animate static textures. In terms of pitch structure, \textit {Viscera} is/will be comprised of long blocks of static harmony. My goal in this work is to create a sense of tumultuous fluctation within these textures. My rhythmic treatment of materials (a wide variety of simultaneous pulse-rates, metric ambiguity) is one component of this. In the absence of activity in the pitch domain, I hope to create a different kind of polyphony and structural development through relationships between physical activities across instruments.

\end{document}
